%!TEX root = ../guide.tex

\section{Inside Components}
\label{sec:2}

This section explains in detail how to define components, create instances of components and connect components.

\subsection{Define a Component}

In TradingSimulation, each component is an Akka\footnote{\url{http://akka.io}} actor. This can be seen from following code snippet, in which the abstract class \emph{Component} indirectly extends the trait \emph{Actor} in Akka. All concrete components extend \emph{Component}.

\begin{lstlisting}[language=Scala]
  trait Receiver extends Actor {
    def receive: PartialFunction[Any, Unit]

    def send[T: ClassTag](t: T): Unit
    def send[T: ClassTag](t: List[T]): Unit
  }

  abstract class Component extends Receiver {
    def start: Unit = {}

    def stop: Unit = {}

    def receiver: PartialFunction[Any, Unit]
  }
\end{lstlisting}

The framework defines three standard methods for every concrete component class:

\begin{itemize}
\item{start}: concrete components can override this method to do custom initialization.
\item{stop}: concrete components can override this method to release resources.
\item{receiver}: concrete components override this method to receive and handle messages.
\end{itemize}

The data transfer between different components is in the form of Akka messages. Concrete component classes have to override the method \emph{receiver} in order to handle the messages they are interested in.

To send a message, a concrete component class can call one of the two \emph{send} methods defined in \emph{Receiver}. The two \emph{send} methods have default implementation in \emph{Component}, a concrete component class should not override them.

Following code snippet illustrates a very simple component which just prints and forwards every message it receives.

\begin{lstlisting}[language=Scala]
  class BackLoop extends Component {
    override def receiver = {
      case m =>
        println(m)
        send(m)
    }
  }
\end{lstlisting}

\subsection{Create Instances of Components}

To create an instance of a component, we have to first create a \emph{component builder} as follows:

\begin{lstlisting}[language=Scala]
val builder = new ComponentBuilder("test")
\end{lstlisting}

Now suppose we have a Component \emph{OhlcIndicator} defined as follows:

\begin{lstlisting}[language=Scala]
  class OhlcIndicator(marketId: Long, symbol: (Currency,Currency), tickSizeMillis: Long) extends Component   {
    // ...
  }
\end{lstlisting}

Then we can create an instance of \emph{OhlcIndicator} with following code snippet:

\begin{lstlisting}[language=Scala]
  val props = Props(classOf[OhlcIndicator], 1L, Currency.USD -> Currency.CHF, 50)
  val ohlc = builder.createRef(props, "ohlc")
\end{lstlisting}

As you can see in the code snippet above, first we create an instance of \emph{Props}\footnote{\url{http://doc.akka.io/api/akka/2.3.1/index.html\#akka.actor.Props}}. The first argument to \emph{Props} is the class we want to instantiate, and the remaining arguments are exactly the parameters that the constructor of the class \emph{OhlcIndicator} expects.

In the second line, \emph{builder.createRef} then takes \emph{props} and a name for the component to create an instance of the component. Note that the return value of \emph{builder.createRef} is a pointer to an instance of \emph{ComponentRef} instead of \emph{OhlcIndicator}.

We have learned how to created instances of components, next let's see how to connect them.

\begin{info}
Internally, \emph{builder.createRef} will call \emph{actorOf} on an instance of \emph{ActorSystem} to create an instance of the component. The whole process seems a little awkward, but that's how Akka works. If you ever have a chance to checkout Akka(\href{http://akka.io}{akka.io}), you'll find it's worth the pain.
\end{info}

\subsection{Connect Components}

To connect an upstream component A to a downstream component B, we need to specify what types of messages B expects to receive from A. During the running A would only send messages of the specified types to B. For example, in the following code snippet, \emph{trader} will only send \emph{LimitBidOrder} messages to \emph{market}, and only send \emph{LimitAskOrder} messages to \emph{display}.

\begin{lstlisting}[language=Scala]
  trader -> (market, classOf[LimitBidOrder])
  trader -> (display, classOf[LimitAskOrder])
\end{lstlisting}

A downstream component can register as many message types as the upstream component can provide. For example, in the following code snippet \emph{trader} would send both \emph{LimitBidOrder} and \emph{LimitAskOrder} messages to \emph{market}.

\begin{lstlisting}[language=Scala]
  trader -> (market, classOf[LimitBidOrder], classOf[LimitAskOrder])
\end{lstlisting}

If a downstream component registers a message type that the upstream component can't provide, the registration has no effect. In practice, it's important to check carefully what messages a component can receive and provide.

\subsection{Traders}

Traders are basically components, but their definition and creation are more involved, so they need a special section.

In order to support automatic optimization of strategy parameters, all traders take parameters as following:

\begin{lstlisting}[language=Scala]
  class MovingAverageTrader(uid: Long, parameters: StrategyParameters) extends Trader(uid, parameters) {
    // ...
  }
\end{lstlisting}

The first parameter is the trader ID, the second parameter is a collection of parameters. There are following types of parameters:

\begin{itemize}
\item \emph{BooleanParameter}: represents boolean parameters.
\item \emph{CoefficientParameter}: represents a floating point coefficient in range [0, 1].
\item \emph{NaturalNumberParameter}: represents an integer number greater or equal to zero.
\item \emph{RealNumberParameter}: represents a real number greater than an initial value.
\item \emph{TimeParameter}: represents a duration of time.
\item \emph{CurrencyPairParameter}: represents a pair of currencies to be traded.
\item \emph{WalletParameter}: represents funds in various currencies.
\item \emph{MarketRulesParameter}: represents rules applied in a certain market.
\end{itemize}

To create an instance of a specific trader, you need to know what parameters it takes exactly. The information about parameters can be found both in code and documentation. Once you know what parameters a trader takes, you can create the \emph{StrategyParameters} object as follows:

\begin{lstlisting}[language=Scala]
  val initialFunds = Map(Currency.CHF -> 5000.0)
  val parameters = new StrategyParameters(
    MovingAverageTrader.INITIAL_FUNDS -> WalletParameter(initialFunds),
    MovingAverageTrader.SYMBOL -> CurrencyPairParameter(symbol),
    MovingAverageTrader.SHORT_PERIOD -> new TimeParameter(periods(0) seconds),
    MovingAverageTrader.LONG_PERIOD -> new TimeParameter(periods(1) seconds),
    MovingAverageTrader.TOLERANCE -> RealNumberParameter(0.0002))
\end{lstlisting}

Now, you can create the trader with just one line of code as follows:

\begin{lstlisting}[language=Scala]
  val trader = MovingAverageTrader.getInstance(10L, parameters, "MovingAverageTrader")
\end{lstlisting}

In the code above, \emph{MovingAverageTrader} is a companion object of the class \emph{MovingAverageTrader}, which provides the method \emph{getInstance} to create a new instance of the component. The method \emph{getInstance} takes a \emph{builder} as implicit parameter. Either make sure an implicit \emph{builder} is in scope or provide it explicitly.

\begin{warning}
In theory, you can also create an instance of \emph{MovingAverageTrader} like a common component as follows. But this is not recommended, as the \emph{getInstance} approach provides more safety check of parameters.

\begin{lstlisting}[language=Scala]
  val trader = builder.createRef(Props(classOf[MovingAverageTrader], 10L, parameters), "trader")
\end{lstlisting}
\end{warning}

Once created, traders can be connected exactly the same way as common components, thus it's omitted here.

Definition of new traders is more involved, which I leave to the interested reader to discover in the code base\footnote{All information about traders can be found in the package \emph{ch.epfl.ts.traders}}.
