\section{Simulation}

\subsection{Simplifying the Real Market}
Our project implements essential parts of the foreign exchange market.
In order to set up a basic market simulator we make the following
simplifications:

\begin{itemize}
    \item traders can only trade one symbol
    \item instruments are restricted to spot transactions
    \item two counterparties are involved: nonfinancial customers and market makers (representing all financial customers)
    \item one market maker per symbol
\end{itemize}

\subsection{Initial Parameters}
Finding appropriate parameters to configure a simulation that reflects the
true market has become a difficult task since existing statistics focus on 
the effects of the market meaning the daily turnover or distribution of market
shares by symbol, trading center or large financial customers. These statistics
are to be seen as results of trading activities. However, this does not allow
us to infer the cause of trading activity. In order to configure a market
simulation in a way that it would reflect the evolution of a real market we 
need to find the cause of activities and therefore we have to answer key questions
such as ``how many people are involved?'' or ``what are their funds?''.

Since we are not able to answer these questions reliably and also since we cannot
instantiate an infinite number of traders we introduce a different concept to
initialize a simulation.

We assign money to traders and market makers in a certain currency (e.g. USD) and 
they receive the equivalent amount in the currency they need to trade their respective
symbol.

\begin{algorithm}
\caption{Distribution of initial funds.}
\label{alg:initfunds}
\begin{algorithmic}[1]
    \Statex Given: distribution of annual income for a specific country
    \Statex
    \ForAll{$s \in symbols$}
        \State  $n \gets$ defined number of nonfinancial $traders$
        \ForAll{$t \in traders(s)$}
            \State $income \gets$ sample of income distribution
            \State $funds(t) \gets income$
        \EndFor
        \State $funds(marketmaker(s)) \gets c \cdot \sum_{i=1}^n funds(traders(i))$
    \EndFor
\end{algorithmic}
\end{algorithm}

%1. define number of non-financial traders
%
%2. for each trader: assign initial funds as 1-year income according to a local income distribution (e.g. Switzerland: http://www.bfs.admin.ch/bfs/portal/en/index/themen/03/04/blank/key/lohnstruktur/lohnverteilung.html)
%
%3. compute market makers initial funds as sum(funds(all traders)) * scale
%

\subsection{Price Definition during Simulation}

When our system is running in pure simulation mode, meaning that ask and bid price is
not defined by historical or live data anymore, prices must also be determined by the
simulation. 

\subsubsection{Balanced Demand and Offer}

\subsubsection{Extreme Situations}

%I might also have found a solution to "how is the price defined when there is nobody on the other side (offering when I want to buy or asking when I want to sell)?".
%here http://www.forextraders.com/market-maker-forex-brokers.html I read that each market maker defines its own spread, which I understand the following way:
%
%if there are only sellers and no buyers:
%-  the ask price is well defined as the lowest ask offer but the bid price is not
%- the market maker defines the bid price by applying its own spread to the lowest ask price
%-> selling at market price means selling to the market maker for the lowest ask price MINUS the spread
%
%if there are only buyers and no sellers:
%- the bid price is well defined as the highest bid offer but the ask price is not
%- again, the market maker defines the bid price by applying its own spread to the highest bid price
%-> buying at market price means buying from the market maker at the highest bid price PLUS the spread
%
%\subsection{Brain Storm}
%
%Market Simulation Research
%
%[1] http://www.tpc.org/tpc_documents_current_versions/pdf/tpce-v1.14.0.pdf
%[2] www.swissquote.com
%[3] http://vantagepointtrading.com/daily-forex-stats
%[4] http://www.bis.org/publ/qtrpdf/r_qt1312e.pdf
%[5] turnovers by counterparty and currency pairs:
%http://www.reuters.com/article/2013/09/05/bis-survey-volumes-idUSL6N0GZ34R20130905
%
%[6] largest forex centers:
%http://countingpips.com/fx/2011/08/8-largest-forex-trading-centers-in-the-world/
%
%[7] directional trading volume:
%http://www.dailyfx.com/forex/technical/article/special_report/2015/04/13/forex-introducing-volume-by-price.html
%
%[8] https://mahifx.com/blog/50-fascinating-facts-about-forex
%
%[9] largest brokers by volume:
%http://www.myfxbook.com/forex-broker-volume
%
%[10] volume survey (North America) with explanation
%http://www.newyorkfed.org/fxc/volumesurvey/explanatory_notes.html
%
%[11] forex glossary
%http://www.cmsfx.com/en/forex-education/Forex-Glossary/ask-price/
%
%[12] salary distribution switzerland
%http://www.bfs.admin.ch/bfs/portal/en/index/themen/03/04/blank/key/lohnstruktur/lohnverteilung.html
%
%[13] explanation of market makers
%http://www.forextraders.com/market-maker-forex-brokers.html
%
%p.47 
%- Entity Relationships
%- Trade Types
%- run historical data for 300 business days before simulation starts
%
%Wanted:
%- trader initial funds -> half of annual income according to [12]?
%- broker initial funds (for leverage trades)
%- market maker funds
%- number of traders per currency pair
%- one market maker per currency pair?

