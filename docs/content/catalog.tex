\section{Catalog of Components}

This section introduces each component that TradingSimulation provides in detail, including what data they expect, what data they output, what parameters can be set.

After reading this section, you should able to select appropriate components for a specific algorithmic trading purpose, parameterize and connect them correctly.

\subsection{Market Simulators}

Market simulators are core components of TradingSimulation, which simulate virtual markets. Market simulators accept ask and bid orders and generate transactions based on \emph{market rules}. In the implementation, market rules are extracted into separate modules to make it possible to define custom market rules and use it in the framework.

Currently TradingSimulation provides two market simulators:

\begin{itemize}
\item OrderBookMarketSimulator
\item MarketFxSimulator
\end{itemize}

\emph{OrderBookMarketSimulator} is usually used to simulate the trading of bitcoins, while \emph{MarketFxSimulator} is used for Forex markets. The Forex market has high liquidity, so it doesn't need order books.

\subsubsection{OrderBookMarketSimulator}

\begin{tabularx}{\textwidth}{|c|X|}
  \hline
  \multicolumn{2}{|c|}{\sc{Parameters}} \\
  \hline
  id: Long  & The market id. It takes one of following values: \emph{MarketNames.BTCE\_ID}, \emph{MarketNames.BITSTAMP\_ID},  \emph{MarketNames.BITFINEX\_ID}. \\
  \hline
  rules: MarketRules & The market rules. The value should be an instance of \emph{MarketRules}, which internally uses order book. \\
  \hline
\end{tabularx}\\[0.4cm]

\noindent
\begin{tabularx}{\textwidth}{|c|X|}
  \hline
  \multicolumn{2}{|c|}{\sc{Message Input}} \\
  \hline
  LimitBidOrder  & the limit bid order from traders or brokers \\
  \hline
  LimitAskOrder  & the limit ask order from traders or brokers \\
  \hline
  MarketBidOrder  & the market bid order from traders or brokers \\
  \hline
  MarketAskOrder  & the market ask order from traders or brokers \\
  \hline
  DelOrder  & the delete order command from traders or brokers \\
  \hline
  PrintBooks  & prints the order book \\
  \hline
\end{tabularx}\\[0.4cm]

\noindent
\begin{tabularx}{\textwidth}{|c|X|}
  \hline
  \multicolumn{2}{|c|}{\sc{Message Output}} \\
  \hline
  Transaction  & the transactions to inform traders or brokers \\
  \hline
  DelOrder  & the delete order to inform traders or brokers \\
  \hline
\end{tabularx}\\[0.4cm]

To create an instance of \emph{OrderBookMarketSimulator}, following code snippet can be used:

\begin{lstlisting}[language=Scala]
val rules = new MarketRules()
val btceMarket = builder.createRef(Props(classOf[OrderBookMarketSimulator], MarketNames.BTCE_ID, rules), MarketNames.BTCE_NAME)
\end{lstlisting}


\subsubsection{MarketFxSimulator}

\begin{tabularx}{\textwidth}{|c|X|}
  \hline
  \multicolumn{2}{|c|}{\sc{Parameters}} \\
  \hline
  id: Long  & The market id. It usually takes the value \emph{MarketNames.FOREX\_ID}. \\
  \hline
  rules: MarketRules & The market rules. The value should be an instance of \emph{ForexMarketRules}. \\
  \hline
\end{tabularx}\\[0.4cm]

\noindent
\begin{tabularx}{\textwidth}{|c|X|}
  \hline
  \multicolumn{2}{|c|}{\sc{Message Input}} \\
  \hline
  LimitBidOrder  & the limit bid order from traders or brokers \\
  \hline
  LimitAskOrder  & the limit ask order from traders or brokers \\
  \hline
  MarketBidOrder  & the market bid order from traders or brokers \\
  \hline
  MarketAskOrder  & the market ask order from traders or brokers \\
  \hline
  Quote  & the quote from live fetcher or playback  \\
  \hline
\end{tabularx}\\[0.4cm]

\noindent
\begin{tabularx}{\textwidth}{|c|X|}
  \hline
  \multicolumn{2}{|c|}{\sc{Message Output}} \\
  \hline
  Transaction  & the transactions to inform traders or brokers \\
  \hline
  Quote  & the quote to inform traders or brokers \\
  \hline
\end{tabularx}\\[0.4cm]

To create an instance of \emph{MarketFxSimulator}, following code snippet can be used:

\begin{lstlisting}[language=Scala]
val rules = new ForexMarketRules()
val forexMarket = builder.createRef(Props(classOf[MarketFXSimulator], MarketNames.FOREX_ID, rules), MarketNames.FOREX_NAME)
\end{lstlisting}

\subsection{Indicators}

Indicators measure the activities of the market. Traders depend on indicators to make sell or buy decisions. Currently TradingSimulation provides following indicators:\\

\noindent
\begin{tabularx}{\textwidth}{|c|X|}
  \hline
  Class & Description  \\
  \hline
  SmaIndicator & Simple Moving Average \\
  \hline
  EmaIndicator & Exponential Moving Average  \\
  \hline
  OhlcIndicator & OHLC(Open-high-low-close)  \\
  \hline
\end{tabularx}\\[0.3cm]

Normally you don't have the need to create indicators, as the traders automatically create indicators that they depend on.

\subsection{Traders}

Traders buy or sell in the virtual market. Different types of traders use different trading strategies. The same strategy may take different parameters thus perform differently. Currently TradingSimulation provides following indicators:\\

\noindent
\begin{tabularx}{\textwidth}{|c|X|}
  \hline
  Class & Description  \\
  \hline
  Arbitrageur & the \emph{arbitrageur} strategy \\
  \hline
  DoubleCrossOverTrader & the \emph{double cross over} strategy  \\
  \hline
  DoubleEnvelopeTrader & the \emph{double envelope} strategy  \\
  \hline
  SobiTrader & the \emph{static order book imbalance} strategy  \\
  \hline
  TransactionVwapTrader & the \emph{volume weighted average price} strategy  \\
  \hline
  SimpleTrader & a simple trader \\
  \hline
  SimpleFXtrader & a simple Forex trader \\
  \hline
\end{tabularx}

\subsection{Fetchers}

Fetchers in TradingSimulation play the role of feeding market data from external sources(internet or file system) into market simulator. Currently TradingSimulation provides following fetchers:\\

\noindent
\begin{tabularx}{\textwidth}{|l|X|}
  \hline
  Class & Description  \\
  \hline
  BtceOrderPullFetcher & the fetcher for BTC-e orders \\
  \hline
  BtceTransactionPullFetcher  & the fetcher for BTC-e transactions  \\
  \hline
  BitfinexOrderPullFetcher & the fetcher for Bitfinex orders  \\
  \hline
  BitfinexTransactionPullFetcher & the fetcher for Bitfinex transactions  \\
  \hline
  BitstampOrderPullFetcher & the fetcher for Bitstamp orders  \\
  \hline
  BitstampTransactionPullFetcher & the fetcher for Bitstamp transactions \\
  \hline
  TrueFxFetcher & the fetcher for  live Forex quotes  \\
  \hline
  HistDataCSVFetcher & reads data from csv source and converts it to Quotes \\
  \hline
\end{tabularx}\\[0.3cm]

Note that all the fetchers in the table above are not components themselves, except \emph{HistDataCSVFetcher}. Non-component fetchers are used together with \emph{PullFetchComponent} as following snippet shows:

\begin{lstlisting}[language=Scala]
  val fetcherFx: TrueFxFetcher = new TrueFxFetcher
  val fetcher = builder.createRef(Props(classOf[PullFetchComponent[Quote]], fetcherFx, implicitly[ClassTag[Quote]]), "TrueFxFetcher")
\end{lstlisting}

\subsection{Persistors}

Persistors load data from and save data to SQLite database. Currently TradingSimulation provides following persistors:

\noindent
\begin{tabularx}{\textwidth}{|l|X|}
  \hline
  Class & Description  \\
  \hline
  OrderPersistor & loads or saves orders from or to database \\
  \hline
  TransactionPersistor  & loads or saves transactions from or to database  \\
  \hline
  QuotePersistor & loads or saves quotes from or to database  \\
  \hline
\end{tabularx}\\[0.3cm]

Note that none of the persistors above are components themselves. They have to be used together with other classes depending one the scenario.

\subsubsection{Storing Data to Database}

To store data to database, the persistors can be used together with the class \emph{Persistor}. The definition of \emph{Persistor} is as follows:

\begin{lstlisting}[language=Scala]
  class Persistor[T: ClassTag](p: Persistance[T]) extends Component {
    // ...
  }
\end{lstlisting}

We can see that the class \emph{Persistor} takes an object of \emph{Persistance} as parameter. Following code illustrates how to use them together:

\begin{lstlisting}[language=Scala]
  // Initialize the Interface to DB
  val btceXactPersit = new TransactionPersistor("btce-transaction-db-batch")
  btceXactPersit.init()

  val btcePersistor = builder.createRef(Props(classOf[Persistor[Transaction]], btceXactPersit, implicitly[ClassTag[Transaction]]), "btcePersistor")
\end{lstlisting}

\subsubsection{Loading Data from Database}

To load data from database, the persistors can be used together with the class \emph{Replay}. The definition of \emph{Replay} is as follows:

\begin{lstlisting}[language=Scala]
  class Replay[T: ClassTag](p: Persistance[T], conf: ReplayConfig) extends Component {
    // ...
  }
\end{lstlisting}

We can see that the class \emph{Replay} takes an object of \emph{Persistance} as parameter. Following code illustrates how to use persistors together with \emph{Replay}:

\begin{lstlisting}[language=Scala]
  // Initialize the Interface to DB
  val btceXactPersit = new TransactionPersistor("btce-transaction-db")
  btceXactPersit.init()

  // Configuration object for Replay
  val replayConf = new ReplayConfig(1418737788400L, 0.01)

  val replayer = builder.createRef(Props(classOf[Replay[Transaction]], btceXactPersit, replayConf, implicitly[ClassTag[Transaction]]), "replayer")
\end{lstlisting}

\subsection{Miscellaneous}

TradingSimulation also provide some utility components as shown in the table below.\\

\noindent
\begin{tabularx}{\textwidth}{|l|X|}
  \hline
  Class & Description  \\
  \hline
  Printer & prints anything it receives in the console \\
  \hline
  Backloop  & saves transaction to database and forward, similar to the UNIX command \emph{tee}  \\
  \hline
  Evaluator & evaluates the performance of traders\\
  \hline
\end{tabularx}\\[0.3cm]
